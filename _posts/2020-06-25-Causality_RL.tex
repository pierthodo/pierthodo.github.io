% Autogenerated translation of 2020-06-25-Causality_RL.md by Texpad
% To stop this file being overwritten during the typeset process, please move or remove this header

\documentclass[12pt]{book}
\usepackage{graphicx}
\usepackage{fontspec}
\usepackage[utf8]{inputenc}
\usepackage[a4paper,left=.5in,right=.5in,top=.3in,bottom=0.3in]{geometry}
\setlength\parindent{0pt}
\setlength{\parskip}{\baselineskip}
\setmainfont{Helvetica Neue}
\usepackage{hyperref}
\pagestyle{plain}
\begin{document}

This blog post will be continually updated until I find its form to be satisfying. I do the same for research paper to force me to get feedback and present ideas early on.

The main question of this post is to study the relationship behind causality and sequential decision making. The driver behind this quest is the fact that most of the literature in both fields relies on the concepts of action and intervention yet rarely(until recently) interacts.  

The first question is why are we interesting in causality for machine learning in the first place. The best explanation I found, was given by Pedro Ortega "Causality gives you generalization".

\end{document}
